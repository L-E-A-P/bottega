\documentclass[a4paper,12pt]{article}
\usepackage[utf8]{inputenc}
\usepackage[T1]{fontenc}
\usepackage[italian]{babel}
\usepackage{AlegreyaSans}

% Attiva AlegreyaSans come font di default
\renewcommand{\familydefault}{\sfdefault}
\usepackage{tikz}
\usetikzlibrary{calc}
\usepackage[left=3cm, right=2cm, top=0.7cm, bottom=0.7cm]{geometry}
\usepackage{graphicx}
\usepackage{tabularx}
\usepackage{array}
\usepackage{multirow}
\usepackage{booktabs}
\usepackage{colortbl}
\usepackage{xcolor}
\usepackage{enumitem}
\usepackage{qrcode}
\usepackage[hidelinks]{hyperref}
\usepackage{url}

% Colori per il report
\definecolor{headcolor}{RGB}{40,40,40}
\definecolor{fieldcolor}{RGB}{245,245,245}
\definecolor{lightgray}{RGB}{220,220,220}

% Font OCR-A a comando
\newfont{\ocrfontsmall}{ocr10 at 8pt}
\newfont{\ocrfontnormal}{ocr10}
\newfont{\ocrfontlarge}{ocr10 at 14pt}
\newfont{\ocrfonthuge}{ocr10 at 18pt}

% Comando semplice per usare OCR-A
\newcommand{\ocr}[1]{{\ocrfontnormal #1}}
\newcommand{\ocrsmall}[1]{{\ocrfontsmall #1}}
\newcommand{\ocrlarge}[1]{{\ocrfontlarge #1}}
\newcommand{\ocrhuge}[1]{{\ocrfonthuge #1}}

% Comando per il codice identificativo verticale
\newcommand{\codiceverticale}[1]{%
    \begin{tikzpicture}[remember picture, overlay]
        \node[rotate=90, anchor=south] at ($(current page.west) + (3cm, 0cm)$) {%
            {\font\tempocr=ocr10 at 81pt \tempocr #1}%
        };
    \end{tikzpicture}%
}

\newcommand{\campo}[2]{%
    \noindent\textbf{#1:} #2\\[0.3em]
}

\newcommand{\campotabella}[2]{%
    \rowcolor{fieldcolor}\textbf{#1} & #2 \\
}

% Impostazioni di pagina per lasciare spazio al codice
\addtolength{\oddsidemargin}{1cm}

\begin{document}
\thispagestyle{empty}

% Codice identificativo del documento (formato: LEAP + GGMMAA + RE + numero incrementale)
\codiceverticale{LEAP190725BR0004}

% Intestazione del report
{\Huge\bfseries\noindent INTERVENTI DI BOTTEGA}\\[0.5em]
{\large LEAP - LABORATORIO ELETTROACUSTICO PERMANENTE, ROMA}\\
19 Luglio 2025

% % Codice univoco del restauro
% \begin{center}
% \fcolorbox{black}{fieldcolor}{\Large\textbf{Codice Restauro: \ocr{LEAP190725RE0001}}}
% \end{center}
% \vspace{1em}

% Sezione 1: Identificazione dello strumento
\section*{IDENTIFICAZIONE}

\begin{tabularx}{\textwidth}{lX}
Nome Strumento & Viola\\
Tipologia & Lira di Eolo \\
Autore & Mario Bertoncini \\
Anno di Costruzione & 1990 \\
Luogo di Costruzione & Cetona, Italia \\
Proprietario & Valeska Bertoncini | Erede e Curatrice del Fondo Mario Bertoncini\\
Deposito & Fondazione Isabella Scelsi \\
Numero di Inventario & \ocr{MB-1990-03} \\
Valore Stimato & € 2000 (in stato dormiente) \\
\end{tabularx}

% Sezione 4: Interventi eseguiti
\section*{INTERVENTI}

\begin{tabularx}{\textwidth}{lX}
  Regolazione meccanica: & Serraggio e orientamento barre metalliche. \\
  Pulizia delle superfici: & Pulizia delle barre metalliche e dei canali di serraggio; pulizia del corpo in legno. \\
  Sostituzione componenti: & %Feltri antiscivolo alla base; bulloneria.\\
\end{tabularx}

% Sezione 4: Interventi eseguiti
\subsection*{MATERIALI}

\begin{tabularx}{\textwidth}{X}
  CRC 5-56 \\
  WD-40 \\
  Sidol \\
  Alcol \\
  Aceto di Alcol \\
  %Bulloni 4,5mm x 40 testa esagonale 8 \\
\end{tabularx}

% Sezione 4: Interventi eseguiti
\subsection*{DURATA}

2 ore

\vfill\null

\begin{flushright}
  \href{https://www.leaphz.net/aleph/index.php/s/rbbZgqLF69snz9j}{
    \qrcode[height=7cm]{https://www.leaphz.net/aleph/index.php/s/rbbZgqLF69snz9j}}
\end{flushright}

\end{document}
