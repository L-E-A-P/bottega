\documentclass[a4paper,12pt]{article}
\usepackage[utf8]{inputenc}
\usepackage[T1]{fontenc}
\usepackage[italian]{babel}
\usepackage{AlegreyaSans}

% Attiva AlegreyaSans come font di default
\renewcommand{\familydefault}{\sfdefault}
\usepackage{tikz}
\usetikzlibrary{calc}
\usepackage[left=3cm, right=2cm, top=0.7cm, bottom=0.7cm]{geometry}
\usepackage{graphicx}
\usepackage{tabularx}
\usepackage{array}
\usepackage{multirow}
\usepackage{booktabs}
\usepackage{colortbl}
\usepackage{xcolor}
\usepackage{enumitem}
\usepackage{qrcode}
\usepackage[hidelinks]{hyperref}
\usepackage{url}
\usepackage{atbegshi}

% Colori per il report
\definecolor{headcolor}{RGB}{40,40,40}
\definecolor{fieldcolor}{RGB}{245,245,245}
\definecolor{lightgray}{RGB}{220,220,220}

% Font OCR-A a comando
\newfont{\ocrfontsmall}{ocr10 at 8pt}
\newfont{\ocrfontnormal}{ocr10}
\newfont{\ocrfontlarge}{ocr10 at 14pt}
\newfont{\ocrfonthuge}{ocr10 at 18pt}

% Comando semplice per usare OCR-A
\newcommand{\ocr}[1]{{\ocrfontnormal #1}}
\newcommand{\ocrsmall}[1]{{\ocrfontsmall #1}}
\newcommand{\ocrlarge}[1]{{\ocrfontlarge #1}}
\newcommand{\ocrhuge}[1]{{\ocrfonthuge #1}}

% Comando per il codice identificativo verticale
\newcommand{\codiceverticale}[1]{%
    \begin{tikzpicture}[remember picture, overlay]
        \node[rotate=90, anchor=south] at ($(current page.west) + (3cm, 0cm)$) {%
            {\font\tempocr=ocr10 at 81pt \tempocr #1}%
        };
    \end{tikzpicture}%
}

% Impostazioni di pagina per lasciare spazio al codice
\addtolength{\oddsidemargin}{1cm}

% Griglia di sfondo per tutte le pagine
\AtBeginShipout{%
    \AtBeginShipoutUpperLeft{%
        \begin{tikzpicture}[remember picture, overlay]
            \draw[lightgray, thin, step=0.5cm] 
                (current page.south west) grid (current page.north east);
        \end{tikzpicture}%
    }%
}

\begin{document}
\thispagestyle{empty}

%% Griglia a pagina intera come sfondo
%\begin{tikzpicture}[remember picture, overlay]
%    \draw[lightgray, very thin, step=0.5cm] 
%        (current page.south west) grid (current page.north east);
%\end{tikzpicture}

% Codice identificativo del documento (formato: LEAP + GGMMAA + RE + numero incrementale)
\codiceverticale{LEAP190725BR0003}

% Intestazione del report
{\Huge\bfseries\noindent NOTE DI BOTTEGA}\\[0.5em]
{\large LEAP - LABORATORIO ELETTROACUSTICO PERMANENTE, ROMA}\\
DATA: 19/07/2025 \hspace{1cm} ORARIO INIZIO: \hspace{2.5cm} ORARIO FINE:\\
\\
\noindent\ocrlarge{MB-1990-02}\\
VIOLINO II - Lira di Eolo\\



% % Codice univoco del restauro
% \begin{center}
% \fcolorbox{black}{fieldcolor}{\Large\textbf{Codice Restauro: \ocr{LEAP190725RE0001}}}
% \end{center}
% \vspace{1em}
\clearpage
\null


\end{document}