\documentclass[a4paper,12pt]{article}
\usepackage[utf8]{inputenc}
\usepackage[T1]{fontenc}
\usepackage[italian]{babel}
\usepackage{AlegreyaSans}

% Attiva AlegreyaSans come font di default
\renewcommand{\familydefault}{\sfdefault}
\usepackage{tikz}
\usetikzlibrary{calc}
\usepackage[left=3cm, right=2cm, top=0.7cm, bottom=0.7cm]{geometry}
\usepackage{graphicx}
\usepackage{tabularx}
\usepackage{array}
\usepackage{multirow}
\usepackage{booktabs}
\usepackage{colortbl}
\usepackage{xcolor}
\usepackage{enumitem}
\usepackage{qrcode}
\usepackage[hidelinks]{hyperref}
\usepackage{url}

% Colori per il report
\definecolor{headcolor}{RGB}{40,40,40}
\definecolor{fieldcolor}{RGB}{245,245,245}
\definecolor{lightgray}{RGB}{220,220,220}

% Font OCR-A a comando
\newfont{\ocrfontsmall}{ocr10 at 8pt}
\newfont{\ocrfontnormal}{ocr10}
\newfont{\ocrfontlarge}{ocr10 at 14pt}
\newfont{\ocrfonthuge}{ocr10 at 18pt}

% Comando semplice per usare OCR-A
\newcommand{\ocr}[1]{{\ocrfontnormal #1}}
\newcommand{\ocrsmall}[1]{{\ocrfontsmall #1}}
\newcommand{\ocrlarge}[1]{{\ocrfontlarge #1}}
\newcommand{\ocrhuge}[1]{{\ocrfonthuge #1}}

% Comando per il codice identificativo verticale
\newcommand{\codiceverticale}[1]{%
    \begin{tikzpicture}[remember picture, overlay]
        \node[rotate=90, anchor=south] at ($(current page.west) + (3cm, 0cm)$) {%
            {\font\tempocr=ocr10 at 81pt \tempocr #1}%
        };
    \end{tikzpicture}%
}

% % Comando per placeholder immagini
% \newcommand{\placeholderimg}[2]{%
%     \fcolorbox{black}{lightgray}{%
%         \begin{minipage}[c][6cm][c]{0.45\textwidth}
%             \centering
%             \large\textbf{[FOTO]}\\
%             \normalsize #1\\[1em]
%             \small #2
%         \end{minipage}
%     }
% }
\newcommand{\campo}[2]{%
    \noindent\textbf{#1:} #2\\[0.3em]
}

\newcommand{\campotabella}[2]{%
    \rowcolor{fieldcolor}\textbf{#1} & #2 \\
}

% Impostazioni di pagina per lasciare spazio al codice
\addtolength{\oddsidemargin}{1cm}

\begin{document}
\thispagestyle{empty}
%\color{magenta}

% Codice identificativo del documento (formato: LEAP + GGMMAA + RE + numero incrementale)
\codiceverticale{LEAP080925BR0009}

% Intestazione del report
{\Huge\bfseries\noindent INTERVENTI DI BOTTEGA}\\[0.5em]
{\large LEAP - LABORATORIO ELETTROACUSTICO PERMANENTE, ROMA}\\
08 Settembre 2025

% % Codice univoco del restauro
% \begin{center}
% \fcolorbox{black}{fieldcolor}{\Large\textbf{Codice Restauro: \ocr{LEAP190725RE0001}}}
% \end{center}
% \vspace{1em}

% Sezione 1: Identificazione dello strumento
\section*{IDENTIFICAZIONE}

\begin{tabularx}{\textwidth}{lX}
Nome Strumento & Kathedrale\\
Tipologia & Gong Eolico \\
Autore & Mario Bertoncini \\
Anno di Costruzione & 1980 \\
Luogo di Costruzione & Berlino, Germania \\
Proprietario & Valeska Bertoncini | Erede e Curatrice del Fondo Mario Bertoncini\\
Deposito & Fondazione Isabella Scelsi \\
Numero di Inventario & \ocr{MB-1980-02} \\
Valore Stimato & € 2000 (in stato dormiente) \\
\end{tabularx}

% Sezione 4: Interventi eseguiti
\section*{INTERVENTI}

\begin{tabularx}{\textwidth}{lX}
  Sostituzioni: & Barre mancanti n. 1 e 35\\
  Pulizia delle superfici: & Pulizia barre smontate una alla volta per rimozione ruggine. \\
\end{tabularx}

% Sezione 4: Interventi eseguiti
\subsection*{MATERIALI}

\begin{tabularx}{\textwidth}{X}
  Barre in acciaio da Fondo Bertoncini\\
  WD-40 \\
  Aceto di Alcol \\
\end{tabularx}

% Sezione 4: Interventi eseguiti
\subsection*{DURATA}

20 ore

\vfill\null

\begin{flushright}
  \href{https://github.com/L-E-A-P/bottega/}{
    \qrcode[height=7cm]{https://github.com/L-E-A-P/bottega/}}
\end{flushright}

\clearpage

\section*{Catalogo delle barre}

Misurazioni ottenute a barre smontate complete di nottolino d'ottone. Le misure delle barre mancanti n. 1 e n. 35 sono state ottenute per continuità della progressione la prima, e interpolazione delle adiacenti per la seconda. La numerazione delle barre è ottenuta partendo dal bullone di ancoraggio al piedistallo.

\begin{table}[h]
\centering
\small
\begin{tabular}{|r|r|r|l||r|r|r|l|}
\hline
\textbf{Barra} & \textbf{L (cm)} & \textbf{Ø (mm)} & \textbf{Note} &
\textbf{Barra} & \textbf{L (cm)} & \textbf{Ø (mm)} & \textbf{Note} \\
\hline
\textbf{1} & \textbf{25,0} & \textbf{3,3} & \textit{stimata} & 27 & 91,0 & 3,6 & \\
2 & 26,1 & 3,3 & & 28 & 82,5 & 3,6 & \\
3 & 27,3 & 3,7 & & 29 & 75,5 & 3,6 & \\
4 & 28,4 & 3,5 & & 30 & 72,4 & 3,6 & \\
5 & 29,4 & 3,3 & & 31 & 56,9 & 3,6 & \\
6 & 30,3 & 3,3 & & 32 & 48,3 & 3,6 & \\
7 & 31,1 & 3,3 & & 33 & 55,3 & 3,6 & \\
8 & 33,1 & 3,3 & & 34 & 63,6 & 3,6 & \\
9 & 34,5 & 3,5 & & \textbf{35} & \textbf{69,5} & \textbf{3,6} & \textit{stimata} \\
10 & 36,6 & 3,3 & & 36 & 75,3 & 3,6 & \\
11 & 42,5 & 3,4 & & 37 & 90,7 & 3,6 & \\
12 & 46,9 & 3,4 & & 38 & 94,4 & 3,6 & \\
13 & 57,1 & 3,5 & & 39 & 84,2 & 3,6 & \\
14 & 50,8 & 3,4 & & 40 & 78,1 & 2,8 & \\
15 & 44,7 & 3,3 & & 41 & 73,6 & 2,8 & \\
16 & 40,1 & 3,2 & & 42 & 70,6 & 2,8 & \\
17 & 34,9 & 3,4 & & 43 & 68,5 & 2,8 & \\
18 & 30,0 & 3,3 & & 44 & 64,1 & 2,8 & \\
19 & 26,6 & 3,4 & & 45 & 61,5 & 2,8 & \\
20 & 33,9 & 3,5 & & 46 & 53,7 & 2,8 & \\
21 & 38,1 & 3,7 & & 47 & 50,4 & 2,8 & \\
22 & 41,9 & 3,7 & & 48 & 46,2 & 2,8 & \\
23 & 47,3 & 3,6 & & 49 & 41,2 & 2,8 & \\
24 & 62,8 & 3,7 & & 50 & 38,3 & 2,8 & \\
25 & 73,0 & 3,6 & & 51 & 32,9 & 2,8 & \\
26 & 98,6 & 3,6 & & 52 & 26,8 & 2,8 & \\
\hline
\end{tabular}
\label{tab:kathedrale}
\end{table}

\null\vfill

\begin{figure}[hb]
\centering
\begin{tikzpicture}[xscale=0.12, yscale=0.051]
% Definizione colori
\definecolor{originale}{RGB}{70,130,180}
\definecolor{stimata}{RGB}{220,20,60}

% Dati barre (numero, lunghezza, tipo: 0=originale, 1=stimata)
\foreach \i/\h/\t in {
1/25.0/1, 2/26.1/0, 3/27.3/0, 4/28.4/0, 5/29.4/0, 6/30.3/0, 7/31.1/0, 
8/33.1/0, 9/34.5/0, 10/36.6/0, 11/42.5/0, 12/46.9/0, 13/57.1/0, 14/50.8/0,
15/44.7/0, 16/40.1/0, 17/34.9/0, 18/30.0/0, 19/26.6/0, 20/33.9/0, 21/38.1/0,
22/41.9/0, 23/47.3/0, 24/62.8/0, 25/73.0/0, 26/98.6/0, 27/91.0/0, 28/82.5/0,
29/75.5/0, 30/72.4/0, 31/56.9/0, 32/48.3/0, 33/55.3/0, 34/63.6/0, 35/69.5/1,
36/75.3/0, 37/90.7/0, 38/94.4/0, 39/84.2/0, 40/78.1/0, 41/73.6/0, 42/70.6/0,
43/68.5/0, 44/64.1/0, 45/61.5/0, 46/53.7/0, 47/50.4/0, 48/46.2/0, 49/41.2/0,
50/38.3/0, 51/32.9/0, 52/26.8/0
} {
  \pgfmathsetmacro{\x}{(\i-1)*2}
  \ifnum\t=1
    \fill[stimata] (\x,0) rectangle ++(1.6,\h);
  \else
    \fill[originale] (\x,0) rectangle ++(1.6,\h);
  \fi
}

% Assi
\draw[->] (0,0) -- (105,0) node[right] {\footnotesize barra};
\draw[->] (0,0) -- (0,105) node[above] {\footnotesize L (cm)};

% Griglia orizzontale
\foreach \y in {25,50,75,100}
  \draw[gray!30] (0,\y) -- (110,\y) node[left,black] {\tiny\y};

% Etichette asse x (ogni 10 barre)
\foreach \x in {1,10,20,30,40,50}
  \draw (\x*2-2,0) -- (\x*2-2,-1) node[below] {\tiny\x};

% Legenda
%\fill[originale] (70,95) rectangle ++(3,3);
%\node[right] at (73.5,96.5) {\footnotesize originali};
%\fill[stimata] (70,88) rectangle ++(3,3);
%\node[right] at (73.5,89.5) {\footnotesize stimate};

\end{tikzpicture}
\label{fig:kathedrale_profile}
\end{figure}
\end{document}
